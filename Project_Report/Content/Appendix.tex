\begin{appendices}
\chapter{C027.cpp}
\begin{lstlisting}
#include "mbed.h"
#include "C027.h"
#include "rtos.h"

int power= 0;
char order= 'O';
int BAUDRATE=9600;
int timer;
int MEASURE_TIME= 1000000000;
C027 devkit;
DigitalOut trigger(D0);

void serialhandler(){
    //devkit.gpsPower(true);
    
    // open the gps serial port
    Serial gps(GPSTXD, GPSRXD);
    gps.baud(BAUDRATE);
    
    // open the PC serial port and (use the same baudrate)
    Serial pc(USBTX, USBRX);
    pc.baud(BAUDRATE);
    //devkit.gpsPower(true);
    while(1){
        if(pc.readable()){
            order = pc.getc();
            if(order=='D' && power){
                devkit.gpsPower(false);
                timer= MEASURE_TIME;
                power = 0;
                }
            else if(order=='E' && !power){
                devkit.gpsPower(true);
                timer= MEASURE_TIME;
                power= 1;
                }
            else if(order=='R'){
                    devkit.gpsReset();
                    timer = 0;
                    }  
        }
    if(gps.readable() && pc.writeable()){
        pc.putc(gps.getc());
        }
    }
}
void measure(){
     while(1){
     trigger.write(1);
     wait_ms(10);
     trigger.write(0); //Trekker utgangen ned
     wait_ms(10);
     timer --;
     while(timer<=0){
         wait_ms(1);
         }
    }
}
int main(){
    devkit.gpsPower(false);
    Thread thread_serial;
    Thread thread_measure;
    
    thread_measure.start(measure);
    thread_serial.start(serialhandler);
   while(1){}
    return 0; }
\end{lstlisting}
\label{Appendix:C027.Cpp}

\chapter{EnergyMeasure.py}
\begin{lstlisting}

# -*- coding: utf-8 -*-
import math
import time
import inspect
import numpy as np
from picoscope import ps6000
from matplotlib.mlab import find
import pylab as pl
import xlwt
import argparse
import decimal


class energyMeasure():
	def __init__(self):
		self.ps = ps6000.PS6000(connect=False)
		self.captureLength = CLENGTH * 1E-3
		self.samplingfreq = SAMPLINGFREQ
		self.capturesampleNo = self.captureLength * (self.samplingfreq * 1E6)
		self.containerA= []
		self.containerB = []
	
	def openScope(self):
		self.ps.open()

		self.ps.setChannel("A", coupling="DC", VRange=1, probeAttenuation=10)
		self.ps.setChannel("B", coupling = "DC", VRange = 5 , probeAttenuation=10)
		self.ps.setChannel("C", enabled= False)
		self.ps.setChannel("D", enabled=False)
		res = self.ps.setSamplingFrequency(self.samplingfreq * 1E6,int(self.capturesampleNo))

		self.sampleRate = res[0]
		print("Sampling @ %f MHz, %d samples"%(res[0]/1E6, res[1]))
		#Use external trigger to mark when we sample
		self.ps.setSimpleTrigger(trigSrc="B",enabled=False)

	def closeScope(self):
		self.ps.close()
	def armMeasure(self):
		self.ps.runBlock()


	def measure(self, filename):
		#setting the maximum number of waveform = 3000 -> 5min*60s*1000*ms*100= 300 000 waveforms
		i=0
		while(1):
			self.armMeasure()
			while(self.ps.isReady() == False):pass
			dataA = self.ps.getDataV("A", int(self.capturesampleNo))
			dataB = self.ps.getDataV("B", int(self.capturesampleNo))
			self.containerA.append(-dataA)
			self.containerB.append(dataB)
			i=i+1

	def plotformat(self):
		self.containerA = np.asarray(self.containerA)
		self.containerB = np.asarray(self.containerB)
		print("The measurments contains:" + str(len(self.containerA))+"waveforms")                                                   
		fig	= pl.figure(figsize=(20,5))
		
		for i in range(len(self.containerA)):
			self.containerB[i][self.containerB[i] < 0.15] = 0
			print("Working on plotting waveform "+str(i)+"of"+str(len(self.containerA)-1))  
			ax 	= 	fig.add_subplot(1,1,1)

			# major ticks every 15, minor ticks every 5
			if(np.amax(self.containerB[i])>0.15):
				ymajor_ticks = np.arange(0, 5, 1)
				yminor_ticks = np.arange(0, 5, 0.25)
			else:
				ymajor_ticks = np.arange(0, 0.500, 0.05)                                              
				yminor_ticks = np.arange(0, 0.500, 0.020)
			
			xmajor_ticks = np.arange(0, 500, 10)                                              
			xminor_ticks = np.arange(0, 500, 5)

			ax.set_xticks(xmajor_ticks)                                                       
			ax.set_xticks(xminor_ticks, minor=True)                                           
			ax.set_yticks(ymajor_ticks)                                                       
			ax.set_yticks(yminor_ticks, minor=True)
			ax.grid(which='minor', alpha=0.2)                                                
			ax.grid(which='major', alpha=0.5)   
			pl.suptitle('Waveform ' +str(i+1), fontsize = 12)
			ax.set_xlabel('Sample number of 100 ms')
			ax.set_ylabel('Voltage')
			pl.plot(self.containerA[i], linewidth= 0.5)
			pl.plot(self.containerB[i], linewidth= 0.5)
			pl.rc('grid', linestyle="-", color='black')
			pl.savefig("log\currentconsumption"+str(i)+args.experimentName+".png")
			#time.sleep(3)
			pl.clf()
		pl.close()
                                                 
	def avgcurrent(self):
		avgcur= []
		bcur = []
		avgpow= []
		temp_c= 0
		temp_b = 0
		temp_p = 0
		
		for i in range(len(self.containerA)):
			temp_c	=	np.average(self.containerA[i])
			temp_b	=	np.amax(self.containerB[i])
			temp_p	=	temp_c*temp_c
			avgcur.append(temp_c)
			bcur.append(temp_b)
			avgpow.append(temp_p)
		
		book = xlwt.Workbook()
		sh = book.add_sheet("Sheet 1")
		style = xlwt.XFStyle()
		# font
		font = xlwt.Font()
		font.bold = True
		style.font = font
		sh.write(0,0,"Name of experiment",style=style)
		sh.write(0,1,args.experimentName,style=style)
		sh.write(1,0,"Waveform",style=style)
		sh.write(1,1,"AvgCurrent(A)",style=style)
		sh.write(1,2,"AvgPower(W)",style=style)
		sh.write(1,3,"AvgCurrentB",style=style)
		
		print("Making excel document")
		for i in range(len(avgcur)):
			sh.write(2+i,0,i+1)
			sh.write(2+i,1,avgcur[i])
			sh.write(2+i,2,avgpow[i])
			sh.write(2+i,3,bcur[i])
		book.save('avgcurrent'+args.experimentName+".xls")



if __name__ == "__main__":


	parser = argparse.ArgumentParser(description='Get statistics.')
	parser.add_argument('-e', dest='experimentName', type=str,required=True, help='Name of the experiment')
	parser.add_argument('-F',dest='samplingFreq', type=float, required=True, help='Sampling frequency in MS/s.')
	parser.add_argument('-c', dest='captureLen', type=float, required=True, help='Capture duration of each waveform in msec')

	args = parser.parse_args()
	SAMPLINGFREQ = args.samplingFreq
	FILENAME = "excel\ " + args.experimentName + ".xls"
	CLENGTH = args.captureLen
	em = energyMeasure()
	em.openScope()
	try:
			start= time.time()
			em.measure(args.experimentName)
			end= time.time()
			print("Execution time=",end-start)
			em.plotformat()
			em.avgcurrent()
		
	except KeyboardInterrupt:
		end= time.time()
		print("Execution time=",end-start)
		em.plotformat()
		em.avgcurrent()
		pass
	em.closeScope()

	#python energymeasure.py -e idag -t 0 -s 1 -F 0.005 -v 1 -c 100
\end{lstlisting}
\label{Appendix:EnergyMeasure.py}
\chapter{L76GNSS.py}
\begin{lstlisting}
from machine import Timer
import time
import gc
import binascii

class L76GNSS:
    STANDBY = bytes( [0x24,0x50,0x4D,0x54,0x4B,0x31,0x36,0x31,0x2C,0x30
    ,0x2A,0x32,0x38,0xD,0xA])
    GLONASS = bytes( [0x24,0x50,0x4D,0x54,0x4B,0x33,0x35,0x33,0x2C,
    0x30,0x2C,0x31,0x2A,0x33,0x36,0xD,0xA]) 
    COLD_START = bytes( [0x24,0x50,0x4D,0x54,0x4B,0x31,0x30,0x34,0x2A,
    0x33,0x37,0xD,0xA])
    PERIODIC_MODE = bytes( [0x24,0x50,0x4D,0x54,0x4B,0x32,0x32,0x35,
    0x2C,0x32,0x2C,0x33,0x30,0x30,0x30,0x2C,0x31,0x32,
    0x30,0x30,0x30,0x2C,0x31,0x38,0x30,0x30,0x30,0x2C,
    0x37,0x32,0x30,0x30,0x30,0x2A,0x31,0x35])

    GPS_I2CADDR = const(0x10)

    def __init__(self, pytrack=None, sda='P22', scl='P21', timeout=None):
        if pytrack is not None:
            self.i2c = pytrack.i2c
        else:
            from machine import I2C
            self.i2c = I2C(0, mode=I2C.MASTER, pins=(sda, scl))

        self.chrono = Timer.Chrono()

        self.timeout = timeout
        self.timeout_status = True

        self.reg = bytearray(1)
        self.i2c.writeto(GPS_I2CADDR, self.reg)
        self.fix = 0
        self.first_fix=0
        self.timestamp= 0
        self.gpgga_s= ''

        self.lat_d = 0
        self.lon_d = 0

    def write_gps(self,data,wait=True):
        print(data)
        self.i2c.writeto(GPS_I2CADDR, data)
        if wait:
             self.wait_gps()    
    def wait_gps(self):
        count = 0
        time.sleep_us(10)
        while self.i2c.readfrom(GPS_I2CADDR, 1)[0] != 0xFF:
            time.sleep_us(100)
            count += 1
            if (count > 500):  # timeout after 50ms
                raise Exception('Pytrack board timeout')


           
    def _read(self):
        self.reg = self.i2c.readfrom(GPS_I2CADDR, 64)
        return self.reg
    def _set_time(self):
        print('_set_time')
        self.timestamp = self.gpgga_s[1]
        print('timestamp set',self.timestamp)

    def _convert_coords(self):
        lat = self.gpgga_s[1]
        lat_d = (float(lat) // 100) + ((float(lat) % 100) / 60)
        lon = self.gpgga_s[3]
        lon_d = (float(lon) // 100) + ((float(lon) % 100) / 60)
        if self.gpgga_s[2] == 'S':
            lat_d *= -1
        if self.gpgga_s[4] == 'W':
            lon_d *= -1
        return(lat_d, lon_d)

    def get_fix(self):
        temp_fix= self.gpgga_s[6]
        self.fix = int(temp_fix)

    def coordinates(self, debug=False):
        lat_d, lon_d, debug_timeout = None, None, False
        if self.timeout != None:
            self.chrono.reset()
            self.chrono.start()
        nmea = b''
        while True:
            if self.timeout != None and self.chrono.read() 
            >= self.timeout:
                self.chrono.stop()
                chrono_timeout = self.chrono.read()
                self.chrono.reset()
                self.timeout_status = False
                debug_timeout = True
            if self.timeout_status != True:
                gc.collect()
                break
            nmea += self._read().lstrip(b'\n\n').rstrip(b'\n\n')
            gpgga_idx = nmea.find(b'GPGGA')
            if gpgga_idx >= 0:
                gpgga = nmea[gpgga_idx:]
                e_idx = gpgga.find(b'\r\n')
                if e_idx >= 0:
                    try:
                        gpgga = gpgga[:e_idx].decode('ascii')
                        print (gpgga)
                        self.gpgga_s = gpgga.split(',')
                        print(self.gpgga_s)
                        self.get_fix()
                        if(self.fix >0):
                            self.lat_d, self.lon_d = self
                            ._convert_coords(self.gpgga_s)
                    except Exception:
                        pass
                    finally:
                        nmea = nmea[(gpgga_idx + e_idx):]
                        gc.collect()
                        break
            else:
                gc.collect()
                if len(nmea) > 4096:
                    nmea = b''
           # time.sleep(0.1)
        self.timeout_status = True
        if debug and debug_timeout:
            print('GPS timed out after %f seconds' % (chrono_timeout))
            return(None, None)
        else:
            return(lat_d, lon_d)


\end{lstlisting}
\label{Appendix:L76GNSS.py}
\chapter{makEnergyModel.py}
\begin{lstlisting}
import xlwt
def make_energy_model():
    fix_period=[1,10,60,1800,3600,14399,14400,15551700,15552000]
    t= [60,3600,86400,2592000,31536000]
    v_sleep, T_sleep, v_wake,T_wake,v_acq, T_acq, v_track, T_track = 3.2*1E-3,0,101*1E-3,5,80*1E-3,0,72*1E-3,1

    P_sleep = v_sleep*v_sleep
    P_wake  = v_wake*v_wake
    P_acq   = v_acq*v_acq
    P_track = v_track*v_track

    columns,rows= len(fix_period),len(t)
    Energy_consumption = [[0 for x in range(columns)] for y in range(rows)]

    i_iterator= 0
    j_iterator= 0

    for i in t:
        for j in fix_period:
            if(j_iterator>columns-1):
                j_iterator = 0   
                if((j<5) and (j<i)):
                    Energy_consumption[i_iterator][j_iterator] = P_track*i
                else:
                    print(i_iterator,j_iterator)
                    Energy_consumption[i_iterator][j_iterator]= -1
                    j_iterator = j_iterator +1
                    continue
            elif(j<14400):
                T_acq   = 1
            elif (j>14399 and j<15552000):
                T_acq   = 30      
            elif(j == 15552000):
                T_acq   = 35
            T_sleep     = j - T_wake - T_acq - T_track 
            Energy_consumption[i_iterator][j_iterator] = (P_sleep*T_sleep + P_wake*T_wake + P_acq*T_acq + P_track*T_track)*i/j
            j_iterator = j_iterator +1

        i_iterator = i_iterator +1
        if(i_iterator>rows-1):
            i_iterator = 0  
    book = xlwt.Workbook()
    sh = book.add_sheet("Sheet 1")
    style = xlwt.XFStyle()
    # font
    font = xlwt.Font()
    font.bold = True
    style.font = font
    
    for i in range(columns):
        for j in range(rows):
            sh.write(i,j,Energy_consumption[j][i])
    book.save('dataa.xls')

    temp    =   Energy_consumption[4][5]
    optimal     =   Energy_consumption[4][6]
    fix_o = 14400

    while(temp<optimal):  
        T_sleep = fix_o - T_wake - 30 - T_track
        optimal = (P_sleep*T_sleep + P_wake*T_wake + P_acq*T_acq + P_track*T_track)*t[4]/fix_o
        print(optimal)
        fix_o = fix_o + 1

    print("Limit for optimizing:",fix_o)

\end{lstlisting}
\label{Appendix:make_energy_model.py}
\end{appendices}