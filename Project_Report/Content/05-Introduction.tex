%%%%%%%%%%%%%%%%%%%%%%%INNLEDNING%%%%%%%%%%%%%%%%%%%%%%%%%%%%%%%%%%%%%%%%%%%%%%%%%%%%%%%%
%%%%%%This is normally Chapter 1 and gives an overview of the assignment and why this work is important. If you have chosen to focus mainly on a part of the assignment text, you may write something about this here and explain why. You also normally give a short description of the structure of the rest of the report towards the end of this chapter.  It is important to indicate which parts that are based on your own work. You may even include a list of your main contributions. %%%%%

\chapter{Introduction}
\pagenumbering{arabic}


%%%%%Introduksjon, GPS i IOT verden osv%%%%%%%

Internet of things (IoT) is the network which consists of objects that are connected to the internet such as sensors, vehicles, actuators and other embedded devices. The objects use the internet to deliver data or to be controlled remotely throughout the Internet. IoT is used in the industry to improve the efficiency of operations, safety and security and give valuable insights in analyzing Big data. The use cases for IoT is also relevant for the general consumer as more of the objects in the household are connected to the Internet. Two examples are: controlling the oven temperature with a mobile app or notifying the hospital if a irregular heart rhythm is noticed by the person's pacemaker. IoT is one of the fastest growing trends in technology today, which is highlighted by Gartners Hype cycle of 2017 shown in figure \ref{fig:Gartner}\\


\begin{minipage}[t]{0.8\textwidth}
    \centering
    \includegraphics[width=0.8\textwidth]{Images/Gartner.PNG}\\
    \captionof{figure}{Gartners Hype Cycle 2017}
    \label{fig:Gartner}
\end{minipage}
\\



\subsection{Problem description}
New challenges has emerged as the development of IoT continues to grow, according to \cite{frank} some of them are:

\begin{enumerate}
    \item The scale in terms of units
    \item The constraints in terms of resources: energy, memory, computation
    \item The non-stationary and heterogeneous environments of things.
    
\end{enumerate}

This project will attempt to solve some of the energy constraint IoT designers face today. The project shall explain and build a model of the energy usage of an IoT device which uses GPS. The model shall be so complex that an IoT application can predict the energy cost of getting a positional fix and determine if an alternative method of getting the positional fix is a better substitute. The analyses of the energy usage should also highlight which parameters that are important to include in the energy model. %%%DEVLEOP THIS
\\
\subsection{Motivation}
\cite{frank} also mentions the opportunities in IoT: 
\begin{enumerate}
    \item Many applications only need to provide an overall picture of a situation.
    \item Nodes can	do energy harvesting and the cloud can support nodes with computation.
    \item There	is much data to learn from.
    
\end{enumerate}
This project is part of a ongoing research project at the Faculty of Information Technology and Electronics at NTNU. The research project is called Autonomous Resource-Constrained Things(ART). The aim of the research project is to develop a method for AI to optimize the IoT infrastructure. The first step in developing this method is to analyze and model the energy consumption of the IoT devices. As part of the ART project Ameen Hussain has proposed and evaluated an energy consumption estimation approach for periodic sensing applications running on the IoT devices
devices\cite{Amen}.As the position of an object is one of the most requested information for IoT applications, the usage of GPS has grown substantially.The GPS is used in embedded systems such as watches, trackers, cellphones and cars. GPS can be one of the most power hungry devices in an IoT system. The power budget for an IoT device is often limited, it is therefore useful to have an energy model of the GPS.\\

The student's motivation for choosing this project was to have some insight and experience regarding the different technologies in IoT. The student thinks the experience will be beneficial later as IoT becomes a dominating trend in the Industry. Another motivation was that the Internet of Things combines multiple fields of technology, and challenges the student to apply all of his knowledge of IT and Electronics.
\newpage

\subsection{Methodology}
WRITE THIS LATER
\subsection{Report Structure}

WRITE THIS LATER 